\documentclass[aspectratio=43]{beamer} %For normal presentation (comment otherwise)
%\documentclass[aspectratio=169]{beamer} %for widescreen prestentation
\usetheme{Marburg}
\usefonttheme{serif}
\usecolortheme{default}%albatross, crane, beetle, dove, fly, seagull, wolverine e beaver.
%%%%%%%%%%%%%%%%%%%%%%%%%%%%%%%%%%%%%%%%%%%%%%%%%%%%%%%%%%%%%%%%%%%%%%%%%%%%%%%%%%%%%%%%%%%%%%%
%%%%%%%%%%%%%%%%%%%%%%%%%%%%%%%%%%%%%%EXTRA PACTAGES%%%%%%%%%%%%%%%%%%%%%%%%%%%%%%%%%%%%%%%%%%%
\usepackage[utf8]{inputenc}
\usepackage[T1]{fontenc}
\usepackage[portuguese, english]{babel}
\usepackage[round]{natbib}
\usepackage{hyperref} 
\usepackage{tcolorbox}
\usepackage{graphicx} % Required for including images
\usepackage{graphics}
\graphicspath{{Imagens/}} % Location of the graphics files
\usepackage{booktabs} % Top and bottom rules for table
\usepackage[font=small,labelfont=bf]{caption}%specifies captions on tables and figures
\usepackage{amsfonts, amsmath, amsthm, amssymb} 
\usepackage{wrapfig} % Allows wrapping text around tables and figures
\usepackage{makeidx}
\usepackage{epstopdf}%adiciona imagens em formato eps no pdf.
\usepackage{subfigure}%cria ambientes de multifiguras
\usepackage{float}%coloca as figuras exatamente aonde você quer
\usepackage{times}
\usepackage{tikz}%pacote para fazer fluxogramas
\usepackage{verbatim}%
\usepackage{multicol}
\usepackage{smartdiagram}
\usepackage[makeroom]{cancel}
\usepackage[framemethod=tikz]{mdframed}
\usepackage{hyperref} 
\usepackage{smartdiagram}
\usesmartdiagramlibrary{additions}
\usepackage{booktabs} % Top and bottom rules for table
\usepackage[font=small,labelfont=bf]{caption} 
\usepackage{subfigure}%cria ambientes de multifiguras


\usepackage{verbatim}%
%%%%%%%%%%%%%%%%%%%%%%%%%%%%%%%%%%%%%%%%%%%%%%%%%%%%%%%%%%%%%%%%%%%%%%%%%%%%%%%%%%%%%%%%%%%%%
%%%%%%%%%%%%%%%%%%%%%%%%%%%%%%%%%%%%%PREAMBLE%%%%%%%%%%%%%%%%%%%%%%%%%%%%%%%%%%%%%%%%%%%%%%%%
\author[Carreira,V.R.]{
Curso:  \\ 
Código:  \\ 
Carga horária: \\
Créditos: \\
Professor: \\
Semestre letivo e ano:  \\
Dias e horários: \\
Local: \\
} 

\title{Nome da disciplina.}
\subtitle{Título tema da aula}
\institute{UFAC - Universidade Federal do Acre}
\date{2025}
\subject{aula}
\setbeamertemplate{footline}[frame number]
\setbeamercovered{transparent}
\setbeamertemplate{navigation symbols}{}
% Tela cheia
\hypersetup{pdfpagemode=FullScreen}
\usepackage{ragged2e}
\justifying
%%%%%%%%%%%%%%%%%%%%%%%%%%%%%%%%%%%%%%%%%%%%%%%%%%%%%%%%%%%%%%%%%%%%%%%%%%%%%%%%%%%%%%%%%%%%%%%%%%%%%%%%%%%%%%%%%%%%%%%%%%%%%%%%%%%%%%PRESENTATION%%%%%%%%%%%%%%%%%%%%%%%%%%%%%%%%%%%%%%%%%%%%%%%%%%%%%%%%%%%%%%%%%%%%%%%%%%%%%%%%%%%%%%%%%%%%%%%%%%%%%%%%%%%%%%%%%%%%%%%%%%%%%%%%%%%%%%%%%%%
\begin{document}

%	\initclock
%	\tdclock
{%\nologo
	\begin{frame}
	%\titlepage
	\begin{center}
    \color{blue}{\huge{Universidade Federal do Acre}}
	\end{center}

	\begin{figure}
		\centering
		\includegraphics[scale=0.125]{Imagens/logoUFAC.png} 
	\end{figure}
\end{frame}
}
\bgroup
\makeatletter
\setbeamertemplate{footline}
\makeatother
\maketitle
\egroup
\scriptsize 
\addtobeamertemplate{navigation symbols}{}{\hskip6pt\raisebox{2pt}{\color{blue}\insertframenumber}}
\setcounter{framenumber}{0}
\AtBeginSection[]
{ \begin{frame}
\centering
\frametitle{Sumário}
\scriptsize
\tableofcontents[currentsection,currentsubsection]% apresenta o sumário antes de cada seção

\end{frame} }

\section{Seção 1}

\section{Seção 2}

\section{Seção 3}


\begin{frame}[allowframebreaks]
%\beamertemplatetextbibitems
\tiny
\bibliographystyle{apalike}
\bibliography{references}
\end{frame}

\makeatother
{
\begin{frame}

	\begin{figure}
		\includegraphics[scale=0.1]{Imagens/logoUFAC.png}
	\end{figure}
	\begin{center}
		\begin{minipage}{0.77\textwidth}
			\small
			\begin{center}
			    Estrada BR-364, Km 04\\
			    CEP 69920-900\\
			    Bairro Distrito Industrial, Rio Branco - AC\\ 
				PABX: 55 68 3901-2580\\
				\url{www.ufac.br}
			\end{center}
		\end{minipage}
	\end{center}
\end{frame}
}



\end{document}